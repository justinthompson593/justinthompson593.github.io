\documentclass{article}
 

 
\usepackage[margin=1in]{geometry} 
\usepackage{amsmath,amsthm,amssymb}
 \usepackage{graphicx}
 \usepackage{enumerate}
 \usepackage{color}
 \usepackage{hyperref}
 
 \hypersetup{urlcolor=cyan}
 
\newcommand{\N}{\mathbb{N}}
\newcommand{\Z}{\mathbb{Z}}

\newcommand\numberthis{\addtocounter{equation}{1}\tag{\theequation}}

\def\R{\mathbb{R}}
\def\Zp{\mathbb{Z}^+}

\def\a{\alpha}
\def\b{\beta}
\def\c{\gamma}

 
\begin{document}
 
% --------------------------------------------------------------
%                         Start here
% --------------------------------------------------------------
 
 
%%%%%%%%%%%%%%%%%%%%%%%%%%%%%%%%%
% TITLE PAGE
%%%%%%%%%%%%%%%%%%%%%%%%%%%%%%%%% 
\title{
    \textmd{\Huge{Midterm B1}}\\
    \textmd{\huge{Section 4}}
}


\maketitle

Consider the set of points $(-2, 2)$, $(1, 1)$ and $(2, 2)$ and the interpolating polynomial $f(x)$, which is a $2^\text{nd}$ degree polynomial that passes through those points. \\

\textbf{Problem 1} [5pt]: Write down $f(x)$ as a Lagrangian polynomial.

Before anything, I do a quick \textit{qualitative} sketch to see what's going on. \hspace*{3cm}\includegraphics[scale=0.5]{thumbSketch}\\

Since we're told that $f$ is a $2^\text{nd}$ degree polynomial, we know that $f$ is either a parabola, a line, or a point. It's clear from my sketch that it has to be a parabola opening upwards. If that's the case, then for $f(x) = \a x^2 + \b x + \c$, we must have $\a$ strictly greater than $0$. It turns out that this quick analysis will not help us to much for \textit{this} problem, but it never hurts to start with an idea of what's going on. That way, if I end up with something like $\a = -2$ in the end, I'll know that I messed up somewhere.




\end{document}











































