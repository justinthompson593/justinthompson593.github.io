\documentclass{article}
 

 
\usepackage[margin=1in]{geometry} 
\usepackage{amsmath,amsthm,amssymb}
 \usepackage{graphicx}
 \usepackage{enumerate}
 \usepackage{color}
 \usepackage{hyperref}
 
 \hypersetup{urlcolor=cyan}
 
\newcommand{\N}{\mathbb{N}}
\newcommand{\Z}{\mathbb{Z}}

\newcommand\numberthis{\addtocounter{equation}{1}\tag{\theequation}}

\def\R{\mathbb{R}}
\def\Zp{\mathbb{Z}^+}

\def\a{\alpha}
\def\b{\beta}
\def\c{\gamma}

 
\begin{document}
 
% --------------------------------------------------------------
%                         Start here
% --------------------------------------------------------------
 
 
%%%%%%%%%%%%%%%%%%%%%%%%%%%%%%%%%
% TITLE PAGE
%%%%%%%%%%%%%%%%%%%%%%%%%%%%%%%%% 
\title{
    \textmd{\Huge{Midterm A1 / A2 / B1 / B2 }}\\
    \textmd{\huge{Section 4 / 2 / 4 / 2}}
}


\maketitle


\textbf{Problem 2 / 3} [10pt]: Finish the following function. For input: $f$ is a function so that $f(x_j)$ returns the value of \textbf{an} interpolating polynomial at $x_j$ and $x$ is a row vector of values. For output: $y$ is a row vector of the values $f(x_j)$ for each element in $x$. Assume that $f(x_j)$ only accepts one value at a time.\\

\textbf{function} y = evaluateF($f$, $x$)\\
\\
\\
\\ 
\\ 

\textbf{end} \\

Okay. This is a \textit{very} tricky problem. When I sat down to take this exam, I almost missed this one completely. In the previous problem, you either (a) found a Lagrangian polynomial named $f$ or (b) you wrote the function $f(x)$ as a linear system. In any case, you've done a bunch of work to accurately describe $f(x)$ and now you're asked to


\end{document}











































